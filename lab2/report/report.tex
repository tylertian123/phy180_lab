% Template source: University of Florida Department of Physics, https://www.phys.ufl.edu/courses/phy4803L/sample-paper.zip

\documentclass[aps,twocolumn,secnumarabic,nobalancelastpage,amsmath,amssymb,nofootinbib,floatfix]{revtex4}

% Documentclass Options
    % aps, prl, rmp stand for American Physical Society, Physical Review Letters, and Reviews of Modern Physics, respectively
    % twocolumn permits two columns, of course
    % nobalancelastpage doesn't attempt to equalize the lengths of the two columns on the last page
        % as might be desired in a journal where articles follow one another closely
    % amsmath and amssymb are necessary for the subequations environment among others
    % secnumarabic identifies sections by number to aid electronic review and commentary.
    % nofootinbib forces footnotes to occur on the page where they are first referenced
        % and not in the bibliography
    % REVTeX 4 is a set of macro packages designed to be used with LaTeX 2e.
        % REVTeX is well-suited for preparing manuscripts for submission to APS journals.


\usepackage{chapterbib}    % allows a bibliography for each chapter (each labguide has it's own)
\usepackage{color}         % produces boxes or entire pages with colored backgrounds
\usepackage{graphics}      % standard graphics specifications
\usepackage[pdftex]{graphicx}      % alternative graphics specifications
\usepackage{epsf}          % old package handles encapsulated post script issues
\usepackage{bm}            % special 'bold-math' package
\usepackage{verbatim}			% for comment environment
\usepackage[colorlinks=true]{hyperref}  % this package should be added after all others
                                        % use as follows: \url{https://urldefense.proofpoint.com/v2/url?u=http-3A__web.mit.edu_8.13&d=DwICAg&c=sJ6xIWYx-zLMB3EPkvcnVg&r=D88uS55Tats-jlFQAC1XryFUYq8B7Lk3StFbXzgsiB4&m=Vjrc9Wj5n5rkIDMPJ5VsRj2GyXC3yXmN_zDHey6dVio&s=_byqsJfgO464rVIugNWFPmbBeIYfNiJcGS1fgIwc0m4&e= }

\usepackage{subcaption}
\usepackage{siunitx}
\usepackage{gensymb}
\usepackage{tikz}
\usepackage{multirow}
\usepackage{xcolor}

% Graph stuff
\usepackage[utf8]{inputenc}
\usepackage{pgfplots}
\usepgfplotslibrary{groupplots,dateplot}
\usetikzlibrary{patterns,shapes.arrows}
\pgfplotsset{compat=newest}
\usepackage{shellesc}
\usetikzlibrary{external}
\tikzexternalize

% Define colours
\definecolor{darkgreen}{HTML}{228833}

\usepackage[english]{babel}
\usepackage[autostyle, english=american]{csquotes}
\MakeOuterQuote{"}

%\addtolength\topmargin{-.5\topmargin} %cuts the top margin in half.

%
% And now, begin the document...
% Students should not have to alter anything above this line
%

\begin{document}
\title{Lab 2: Determining the Relationship Between Period and Amplitude}
\author{Tyler Tian}
\date{\today}


\begin{abstract}
In this experiment, the relationship between the period and amplitude of the pendulum from Lab 1 is determined. The
peaks of the data are found and processed to extract data of period vs. amplitude. A power series is fit to the 
period-amplitude data in order to determine the existence and nature of the relationship between the period and
amplitude.

The final relationship between the period and amplitude is determined to be quadratic in nature:
$T(\theta) = (1.63\si{s} \pm 0.05\si{s}) + (0.111\si{s} \pm 0.003\si{s})\theta^2$ The approximation that the period
is constant ($T = 1.662\si{s}$) is determined to be only valid for $|\theta| < 0.79\si{rad}$ or approximately
$45\degree$.
\end{abstract}

\maketitle

%%%%%%%%%%%%%%%%%%%%%%%%%%%%%%%%%%%%%%%%%%%%%%%%%%%%%%%%%%%%%%%%%%

\section{Introduction}

This experiment seeks to determine whether there exists a relationship between the period and amplitude of a pendulum.
From Lab 1, the damped pendulum is modelled as
\begin{equation}
    \theta(t) = \theta_0 e^{-\frac{t}{\tau}}\cos\left(2\pi\frac{t}{T} + \phi_0\right)
    \label{eqn:model}
\end{equation}
where $\theta(t)$ is the angle of the pendulum in radians at time $t$, $\theta_0$ is the initial amplitude at release,
$\tau$ is the time constant of decay, $T$ is the period and $\phi_0$ is the phase shift.

According to this model, there should be no relationship between the period and amplitude of the pendulum. This lab
seeks to confirm this by collecting amplitude-period data over multiple oscillations and trials, and fitting it to power
series of different degrees:
\begin{equation}
    T(\theta) = T_0 + B\theta + C\theta^2 + D\theta^3 + \cdots
    \label{eqn:power_series}
\end{equation}

Then, the values of $B, C, \cdots$ can be used to determine whether a relationship exists.

%%%%%%%%%%%%%%%%%%%%%%%%%%%%%%%%%%%%%%%%%%%%%%%%%%%%%%%%%%%%%%%%%%

\section{Method}

\subsection{Changes to Data Collection Methods}

As there were no problems identified in the pendulum setup from the last lab, the main structure of the pendulum was
unmodified. However, to accommodate for faster bob velocities, a few changes were made to the rest of the data
collection methods:

\begin{enumerate}
    \item The reflective tape used to marked the pivot was changed into green masking tape. The reflective tape was
          hard to track as it would have a different brightness and colour depending on the viewing and light angle.
          Masking tape allows for more consistent and more accurate tracking due to its uniform colour.
    \item The framerate was increased to 120fps (from 60fps). The larger amplitudes in this lab lead to larger
          velocities and more motion blur, making the tracking less reliable and less accurate. Increasing the framerate
          reduces motion blur and allows for better tracking, leading to smaller time and angle uncertainties.
    \item The light source was changed to a flood light. The previous light source (ceiling light) was not bright or
          focused enough for reliable tracking. A RYOBI ONE+ LED Work Light was used instead of the ceiling light to
          improve tracking.
\end{enumerate}

\begin{figure}[htb]
    \includegraphics[width=0.75\linewidth]{experiment_setup.jpg}
    \caption{The new experiment setup. The red box marks the location of the phone used to record the experiment.}
\end{figure}

The vision tracking program was unmodified (see Appendix \ref{appendix:code}).

\subsection{Data Processing and Fitting}

To convert from raw time-angle data to amplitude-period data required for this lab, a Python program was used to find
the peaks and valleys of the graph, as shown in Figure \ref{fig:rawdata} (see Appendix \ref{appendix:code}).
Since the data was somewhat noisy, multiple peak points were averaged to find the time of each peak, and the
maximum/minimum of these peak points were taken to find the peak amplitude.

Measuring the period by taking the time for multiple oscillations and dividing by the number of oscillations will reduce
the period uncertainty, but increase the amplitude uncertainty as the amplitude decays more over a longer period of
time. Conversely, taking the time for half an oscillation and then multiplying by 2 increases the period uncertainty but
reduces the amplitude uncertainty as there is now less decay.

Because the $Q$ factor determined from Lab 1 experiment is reasonably large (about $140$), the amplitude does not decay
too much between oscillations, so it is not necessary to measure half-oscillations to reduce the amplitude uncertainty.
The final decision was to measure the period directly using the difference in time between adjacent peaks.

\begin{figure}[htb]
    \begin{tikzpicture}
        \begin{axis}[
            title=Collected Data,
            xlabel=Time (s),
            ylabel=Angle (rad),
            legend entries={Raw Data, Min/Max}
        ]
            \addplot+[
                only marks,
                mark size=1pt,
            ] table [x index=0, y index=1] {trial1_rawdata.txt};
            \addplot+[
                only marks,
                mark size=1.5pt,
            ] table [x index=0, y index=1] {trial1_extrema.txt};
        \end{axis}
    \end{tikzpicture}
    \caption{Zoomed-in view of the first 500 raw data points with the peaks marked.}
    \label{fig:rawdata}
\end{figure}

These are the sources of uncertainty resulting from this method:
\begin{enumerate}
    \item Time (period) uncertainty from the camera: The camera can only capture a set number of frames per second and
          has a nonzero shutter speed. The true peak of an oscillation could lie between two data points, creating an
          uncertainty. An upper bound for this uncertainty can be obtained by taking the time between two frames and
          dividing by 2. At 30 data points per second, this results in an uncertainty of \(\pm 0.02\si{s}\).
          (\textit{Note: while the video was recorded in 120fps to reduce motion blur, the actual rate of data
          collection was only 30 data points/second.})
    \item Period uncertainty from noise in the data and inaccuracies in peak finding: Due to noise in the data,
          sometimes the true peak location is not apparent. The code attempts to correct this by taking the mean of all
          peak candidates. The uncertainty of the mean ($\frac{\sigma}{\sqrt{N}}$) of the peak candidates is used as the
          uncertainty, which is computed individually for each data point as it varies.
    \item Amplitude uncertainty caused by the previous two uncertainties: Since the true peak may lie between two data
          points (as per \textcircled{1}) and is influenced by noise (as per \textcircled{2}), this creates an
          uncertainty. As the exact value of this uncertainty is hard to find, it is assumed to be the same in
          percentage as the larger of the two sources above.
    \item Amplitude uncertainty caused by the decay of the pendulum over one oscillation: The pendulum's amplitude
          decays as it swings, creating an uncertainty. By substituting $t = T$ into the pendulum equation, we obtain
          an exponential term of $e^{-\frac{T}{t}} = e^{-\frac{\pi}{Q}}$. In other words, the amplitude decays by a
          factor of $e^{-\frac{\pi}{Q}}$ per oscillation. The relative change in amplitude is then
          $1 - e^{-\frac{\pi}{Q}}$; by substituting in the $Q$ value obtained in the previous lab (about 140), the
          amount of decay is found to be $2.2\%$, so the uncertainty is \(\pm 2\%\).
\end{enumerate}

Additionally, there may be an amplitude uncertainty resulting from imperfect computer vision tracking, but due to the
improved setup, this can be assumed to be much less than the other sources listed above. Plotting the data points and
uncertainties yields Figure \ref{fig:data1}.

\begin{figure}[htb]
    \begin{tikzpicture}
        \begin{axis}[
            title=Period vs. Amplitude,
            xlabel=Amplitude (rad),
            ylabel=Period (s),
            xmin=-2,
            xmax=2,
        ]
            \addplot+[
                only marks,
                mark size=1pt,
                error bars/.cd,
                    x dir=both,
                    y dir=both,
                    x explicit,
                    y explicit,
            ] table [
                x index=0,
                y index=1,
                x error index=2,
                y error index=3,
            ] {data_with_uncertainties.txt};
        \end{axis}
    \end{tikzpicture}
    \caption{Data points collected from one trial, plotted with uncertainty bars.}
    \label{fig:data1}
\end{figure}

After the data is processed, a Python program (see Appendix \ref{appendix:code}) is used to fit power series of various
degrees (Equation \ref{eqn:power_series}) to the data. Results are shown below.

%%%%%%%%%%%%%%%%%%%%%%%%%%%%%%%%%%%%%%%%%%%%%%%%%%%%%%%%%%%%%%%%%%

\section{Observations}

Data points collected from all 5 trials are combined for the fit.
Table \ref{table:fit} shows the optimal parameters resulting from fitting power series of various degrees to the data.
Note that the parameter and uncertainty values in Table \ref{table:fit} are unrounded to allow for later analysis;
refer to later sections for properly rounded values.

The uncertainties in Table \ref{table:fit} take into account both the uncertainties created by the fit and the
uncertainties in the period and amplitude data that carries through, by taking the maximum of the two. The maximum
relative uncertainty for any data point was determined to be 2.95\%.

\begin{table}[h]
    \begin{tabular}{c|c|r|r}
        Degree & Param. & \multicolumn{1}{c|}{Value} & \multicolumn{1}{c}{Max Uncertainty} \\
        \hline
        0                  & $T_0$ & 1.662 & \pm 0.0490 \\
        \hline
        \multirow{2}{*}{1} & $T_0$ & 1.662 & \pm 0.0490 \\
                           & $B$ & -0.00298 & \pm 0.00283 \\
        \hline
        \multirow{3}{*}{2} & $T_0$ & 1.633 & \pm 0.0481 \\
                           & $B$ & 0.00257 & \pm 0.00129 \\
                           & $C$ & 0.1109 & \pm 0.00327 \\
        \hline
        \multirow{4}{*}{3} & $T_0$ & 1.633 & \pm 0.0481 \\
                           & $B$ & 0.00323 & \pm 0.00201 \\
                           & $C$ & 0.1108 & \pm 0.00326 \\
                           & $D$ & -0.000788 & \pm 0.00190 \\
        \hline
        \multirow{5}{*}{4} & $T_0$ & 1.634 & \pm 0.0481 \\
                           & $B$ & 0.00283 & \pm 0.00202 \\
                           & $C$ & 0.1041 & \pm 0.00449 \\
                           & $D$ & -0.000177 & \pm 0.00194 \\
                           & $E$ & 0.00431 & \pm 0.00270 \\
    \end{tabular}
    \caption{Optimal parameters from the fit for various degrees. Note the values and uncertainties are unrounded.}
    \label{table:fit}
\end{table}

Graphs of the fit and residuals are shown for degrees 0 to 2 in Figures \ref{fig:fit} to \ref{fig:residuals2}.
Uncertainty bars are omitted for Figure \ref{fig:fit} to keep the graph readable, since there are too many data points.

\begin{figure*}[t]
    \centering
    \begin{subfigure}{\textwidth}
        \begin{tikzpicture}
            \begin{axis}[
                title=Period vs. Amplitude,
                xlabel=Amplitude (rad),
                ylabel=Period (s),
                xmin=-1.8,
                xmax=1.8,
                xtick distance=0.3,
                legend entries={Collected Data, Degree 0 Fit, Degree 1 Fit, Degree 2 Fit},
                legend style={
                    at={(0.5, 0.98)},
                    anchor=north
                },
                width=\textwidth,
                height=0.5\textwidth,
            ]
                \addplot+[
                    black,
                    mark options={fill=black},
                    only marks,
                    mark size=1pt,
                ] table [x index=0, y index=1] {fit_data.txt};
                \addplot[
                    blue,
                    very thick,
                    domain=-1.6:1.6,
                    samples=3,
                ] {1.662};
                \addplot[
                    red!75!white,
                    very thick,
                    domain=-1.6:1.6,
                    samples=3,
                ] {1.662 - 0.00298*x};
                \addplot[
                    darkgreen,
                    very thick,
                    domain=-1.6:1.6,
                    samples=200,
                ] {1.633 + 0.00257*x + 0.1109*x^2};
            \end{axis}
        \end{tikzpicture}
        \caption{Result of fitting a 0, 1, and 2 degree power series to the data. Error bars are omitted for readability.}
        \label{fig:fit}
    \end{subfigure}
    \begin{subfigure}{0.49\textwidth}
        \begin{tikzpicture}
            \begin{axis}[
                title=Residuals for Degree 0 Fit,
                xlabel=Amplitude (rad),
                ylabel=Period (s),
                extra y ticks={0},
                extra y tick labels={},
                extra y tick style={grid=major}
            ]
                \addplot+[
                    only marks,
                    mark size=1pt,
                    error bars/.cd,
                        x dir=both,
                        y dir=both,
                        x explicit,
                        y explicit,
                ] table [
                    x index=0,
                    y index=1,
                    x error index=2,
                    y error index=3,
                ] {residuals_deg0.txt};
            \end{axis}
        \end{tikzpicture}
        \caption{Residuals from fitting the data to \(T(\theta) = T_0\).}
        \label{fig:residuals0}
    \end{subfigure}
    \begin{subfigure}{0.49\textwidth}
        \begin{tikzpicture}
            \begin{axis}[
                title=Residuals for Degree 2 Fit,
                xlabel=Amplitude (rad),
                ylabel=Period (s),
                extra y ticks={0},
                extra y tick labels={},
                extra y tick style={grid=major},
                yticklabel style={
                    /pgf/number format/precision=3,
                    /pgf/number format/fixed
                }
            ]
                \addplot+[
                    only marks,
                    mark size=1pt,
                    error bars/.cd,
                        x dir=both,
                        y dir=both,
                        x explicit,
                        y explicit,
                ] table [
                    x index=0,
                    y index=1,
                    x error index=2,
                    y error index=3,
                ] {residuals_deg2.txt};
            \end{axis}
        \end{tikzpicture}
        \caption{Residuals from fitting the data to \(T(\theta) = T_0 + B\theta + C\theta^2\).}
        \label{fig:residuals2}
    \end{subfigure}
\end{figure*}

%%%%%%%%%%%%%%%%%%%%%%%%%%%%%%%%%%%%%%%%%%%%%%%%%%%%%%%%%%%%%%%%%%

\section{Analysis and Conclusion}

\subsection{Asymmetry}

Asymmetry in the pendulum is indicated by a nonzero coefficient on the odd-powered terms in the series ($B, D, \cdots$).
This is because the odd-powered terms are the only ones with different behaviour for $\theta < 0$ versus $\theta > 0$,
so these terms being nonzero would indicate different behaviour depending on the side that the bob starts on.

From Table \ref{table:fit}, for all the cases where $B$ is a parameter, its value never exceeds more than 2 times its
uncertainty. Furthermore, for most of the fits, $B$ is about or less than 1.5 times its uncertainty, therefore
\(B\) is consistent with zero. Additionally, for all the cases where $D$ is a parameter, its value is always much less
than its uncertainty, so \(D\) is experimentally zero. As both odd terms are consistent with zero, it can be
concluded that \textbf{there is no asymmetry in the pendulum}.

\subsection{Relationship Between Period and Amplitude}

From simply observing Figure \ref{fig:fit}, it is clear that the data forms a distinct pattern that is not
well-approximated by a constant or linear model of $T(\theta)$. Indeed, the value of $C$, the quadratic term, is always
much larger than its uncertainty (always more than $20 \times$).

By examining the residuals of the degree 0 fit in Figure \ref{fig:residuals0}, it is clear that \(T(\theta) = T_0\) is
not a correct model for the data, as there is a clear parabolic pattern in the residuals, with some points having a
residual more than 5 times their uncertainty. On the other hand, the residuals of the degree 2 fit in Figure
\ref{fig:residuals2} exhibit no clear pattern, especially when \(\theta\) is large.

Note there appears to be a pattern resembling a downward parabola in Figure \ref{fig:residuals2} when \(\theta\) is small.
This can be attributed to the discretization of time due to the data collection method. Because angle data is taken 30
times per second, the period has a resolution of \(1/30\si{s}\) without averaging. This is evident in Figure
\ref{fig:fit} as most data points seem to lie on discrete horizontal lines. Despite the pattern, the residual points are
evenly distributed above and below the zero line, indicating a good fit.

Furthermore, the higher-degree terms $D$ and $E$ are both smaller than 2 times their uncertainties, so they are
consistent with zero. This suggests a trend that no matter how many terms are in the equation, the terms with degree
higher than 2 should all be consistent with zero. Additionally, as $B$ is shown to be consistent with zero in the
previous subsection, it can be concluded that the actual $T(\theta)$ has a form of $T(\theta) = T_0 + C\theta^2$, i.e.
\textbf{there exists a quadratic relationship between period and amplitude}.

Rounding the parameters and uncertainties gives the final equation
\begin{equation}
    T(\theta) = (1.63\si{s} \pm 0.05\si{s}) + (0.111\si{s} \pm 0.003\si{s})\theta^2
\end{equation}

\subsection{Valid Range of Constant Period Approximation}

Since the relationship between period and amplitude is actually quadratic in nature, the approximation that
$T(\theta) = T_0$ (i.e. the period is constant) is only valid for small angles where $\theta^2 \approx 0$.

One way of determining when this approximation stops being valid is to see when the period as predicted by the constant
approximation stops being consistent with the period as predicted by the quadratic model. Here "consistent" is defined
as the values differing by no more than 2 uncertainties. Therefore, the valid range of the approximation is
\begin{equation}
    T(\theta) - T_0 \leq 2\epsilon
\end{equation}
where $T(\theta)$ is the quadratic period prediction, $T_0$ is the constant period approximation and $\epsilon$ is the
period error. As most data points have a period uncertainty of $\pm 0.02\si{s}$, substituting this into the equation creates
\begin{equation}
    1.633 + 0.1109\theta^2 - 1.662 \leq 0.04
\end{equation}

Solving this inequality produces the condition $-0.79 \leq \theta \leq 0.79$. In other words, for this pendulum,
\textbf{the constant period approximation is only valid when the absolute amplitude is no more than $0.79\si{rad}$ or
$45\degree$}.

%%%%%%%%%%%%%%%%%%%%%%%%%%%%%%%%%%%%%%%%%%%%%%%%%%%%%%%%%%%%%%%%%%

\appendix

\section{Source Code}

A comprehensive list of all source code, as well as the \LaTeX{} source for this report, can be found on GitHub at
\url{https://github.com/tylertian123/phys180_lab/tree/lab2/}, in particular:
\label{appendix:code}
\begin{enumerate}
    \item Vision tracking program for collecting raw data: \url{https://github.com/tylertian123/phys180_lab/tree/lab2/gendata.py}
    \item Program for finding the peaks of the raw data and collecting period-amplitude data: \url{https://github.com/tylertian123/phys180_lab/tree/lab2/lab2/process_data.py}
    \item Program for fitting the power series: \url{https://github.com/tylertian123/phys180_lab/tree/lab2/lab2/fit.py}
\end{enumerate}

\end{document}
